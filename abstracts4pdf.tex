% Options for packages loaded elsewhere
\PassOptionsToPackage{unicode}{hyperref}
\PassOptionsToPackage{hyphens}{url}
\PassOptionsToPackage{dvipsnames,svgnames,x11names}{xcolor}
%
\documentclass[
  letterpaper,
  DIV=11,
  numbers=noendperiod]{scrartcl}

\usepackage{amsmath,amssymb}
\usepackage{iftex}
\ifPDFTeX
  \usepackage[T1]{fontenc}
  \usepackage[utf8]{inputenc}
  \usepackage{textcomp} % provide euro and other symbols
\else % if luatex or xetex
  \usepackage{unicode-math}
  \defaultfontfeatures{Scale=MatchLowercase}
  \defaultfontfeatures[\rmfamily]{Ligatures=TeX,Scale=1}
\fi
\usepackage{lmodern}
\ifPDFTeX\else  
    % xetex/luatex font selection
\fi
% Use upquote if available, for straight quotes in verbatim environments
\IfFileExists{upquote.sty}{\usepackage{upquote}}{}
\IfFileExists{microtype.sty}{% use microtype if available
  \usepackage[]{microtype}
  \UseMicrotypeSet[protrusion]{basicmath} % disable protrusion for tt fonts
}{}
\makeatletter
\@ifundefined{KOMAClassName}{% if non-KOMA class
  \IfFileExists{parskip.sty}{%
    \usepackage{parskip}
  }{% else
    \setlength{\parindent}{0pt}
    \setlength{\parskip}{6pt plus 2pt minus 1pt}}
}{% if KOMA class
  \KOMAoptions{parskip=half}}
\makeatother
\usepackage{xcolor}
\setlength{\emergencystretch}{3em} % prevent overfull lines
\setcounter{secnumdepth}{-\maxdimen} % remove section numbering
% Make \paragraph and \subparagraph free-standing
\makeatletter
\ifx\paragraph\undefined\else
  \let\oldparagraph\paragraph
  \renewcommand{\paragraph}{
    \@ifstar
      \xxxParagraphStar
      \xxxParagraphNoStar
  }
  \newcommand{\xxxParagraphStar}[1]{\oldparagraph*{#1}\mbox{}}
  \newcommand{\xxxParagraphNoStar}[1]{\oldparagraph{#1}\mbox{}}
\fi
\ifx\subparagraph\undefined\else
  \let\oldsubparagraph\subparagraph
  \renewcommand{\subparagraph}{
    \@ifstar
      \xxxSubParagraphStar
      \xxxSubParagraphNoStar
  }
  \newcommand{\xxxSubParagraphStar}[1]{\oldsubparagraph*{#1}\mbox{}}
  \newcommand{\xxxSubParagraphNoStar}[1]{\oldsubparagraph{#1}\mbox{}}
\fi
\makeatother


\providecommand{\tightlist}{%
  \setlength{\itemsep}{0pt}\setlength{\parskip}{0pt}}\usepackage{longtable,booktabs,array}
\usepackage{calc} % for calculating minipage widths
% Correct order of tables after \paragraph or \subparagraph
\usepackage{etoolbox}
\makeatletter
\patchcmd\longtable{\par}{\if@noskipsec\mbox{}\fi\par}{}{}
\makeatother
% Allow footnotes in longtable head/foot
\IfFileExists{footnotehyper.sty}{\usepackage{footnotehyper}}{\usepackage{footnote}}
\makesavenoteenv{longtable}
\usepackage{graphicx}
\makeatletter
\def\maxwidth{\ifdim\Gin@nat@width>\linewidth\linewidth\else\Gin@nat@width\fi}
\def\maxheight{\ifdim\Gin@nat@height>\textheight\textheight\else\Gin@nat@height\fi}
\makeatother
% Scale images if necessary, so that they will not overflow the page
% margins by default, and it is still possible to overwrite the defaults
% using explicit options in \includegraphics[width, height, ...]{}
\setkeys{Gin}{width=\maxwidth,height=\maxheight,keepaspectratio}
% Set default figure placement to htbp
\makeatletter
\def\fps@figure{htbp}
\makeatother

\KOMAoption{captions}{tableheading}
\makeatletter
\@ifpackageloaded{caption}{}{\usepackage{caption}}
\AtBeginDocument{%
\ifdefined\contentsname
  \renewcommand*\contentsname{Table of contents}
\else
  \newcommand\contentsname{Table of contents}
\fi
\ifdefined\listfigurename
  \renewcommand*\listfigurename{List of Figures}
\else
  \newcommand\listfigurename{List of Figures}
\fi
\ifdefined\listtablename
  \renewcommand*\listtablename{List of Tables}
\else
  \newcommand\listtablename{List of Tables}
\fi
\ifdefined\figurename
  \renewcommand*\figurename{Figure}
\else
  \newcommand\figurename{Figure}
\fi
\ifdefined\tablename
  \renewcommand*\tablename{Table}
\else
  \newcommand\tablename{Table}
\fi
}
\@ifpackageloaded{float}{}{\usepackage{float}}
\floatstyle{ruled}
\@ifundefined{c@chapter}{\newfloat{codelisting}{h}{lop}}{\newfloat{codelisting}{h}{lop}[chapter]}
\floatname{codelisting}{Listing}
\newcommand*\listoflistings{\listof{codelisting}{List of Listings}}
\makeatother
\makeatletter
\makeatother
\makeatletter
\@ifpackageloaded{caption}{}{\usepackage{caption}}
\@ifpackageloaded{subcaption}{}{\usepackage{subcaption}}
\makeatother
\ifLuaTeX
  \usepackage{selnolig}  % disable illegal ligatures
\fi
\usepackage{bookmark}

\IfFileExists{xurl.sty}{\usepackage{xurl}}{} % add URL line breaks if available
\urlstyle{same} % disable monospaced font for URLs
\hypersetup{
  colorlinks=true,
  linkcolor={blue},
  filecolor={Maroon},
  citecolor={Blue},
  urlcolor={Blue},
  pdfcreator={LaTeX via pandoc}}

\author{}
\date{}

\begin{document}

Blockchain-as-a-Service for Business Process Management: Survey and
Challenges (@viriyasitavatBlockchainasaServiceBusinessProcess2023)

Blockchain technology (BCT) has brought a paradigm shift to Business
Process Management (\hl{BPM}). BCT provides a trusted decentralized
infrastructure to secure data and process executions using distributed
ledgers and smart contract to manage complex business processes.
Numerous efforts have been made to exploit BCT in supporting dynamic and
trusted collaborations of business processes. This paper aims to
understand recent BCT development for its \hl{BPM} applications and
identify the limitations and challenges for further development via a
systematic literature review (SLR). It is found that numerous works have
reported using BCT as technical solutions to fulfill some traditional
\hl{BPM} functions. This paper is distinguished from existing works,
especially several relevant surveys in the sense that (1) the impact of
using BCT in \hl{BPM} is thoroughly explored to identify new constraints
and challenges explicitly brought by blockchains; (2) the requirements
for Business Process Compliance (BPC) are firstly analyzed in detail.
Note that BPC is to assure the adherence of business processes to
pre-defined policies, standards, specifications, regulations, and laws
when business processes are executed. To fill the gaps of BCT
applications in these two aspects, Blockchain-as-aService (BCaaS) is
adopted in business process architecture, and the trends of BCT
developments are identified accordingly.

Keywords: No keywords available

Toward Agile Business Process Management: Description of Concepts and a
Proposed Definition (@bernardojuniorAgileBusinessProcess2023)

Business Process Management (\hl{BPM}) needs to be adjusted quickly and
flexibly to cope with the dynamics of the business environment, so the
demand for the incorporation of agility has reached \hl{BPM}. To
contribute to the theoretical consolidation of Agile \hl{BPM}, it is
necessary to develop a conceptualization for the term, that is, to
describe the essential attributes for its understanding. Communicating
the meaning of the concept in reduced words occurs through the
definition, so the main objective of this study is to develop a
scientific definition for Agile \hl{BPM}. This study was performed in
three phases. First, a systematic literature review was conducted to
investigate how the scientific literature has addressed Agile \hl{BPM}.
Next, a deductive analysis was performed to conceptualize Agile
\hl{BPM}. In the third phase, a consultation with experts was conducted
to refine the conceptual view and critique a tentative definition,
preceded by judges' analysis to consolidate the definition. As a result,
the concept of Agile \hl{BPM} was elaborated, and based on the reduction
of this conceptualization, a scientific definition was presented which
describes that Agile \hl{BPM} is ``the promotion of \hl{BPM} in which
practitioners stimulate change quickly and flexibly in order to meet
organizational demands with compliance and provide a better customer
experience'\,'. There is a pioneering spirit in the present study
regarding the deliberate conceptualization of Agile \hl{BPM}, which
provides the basis for discussion of the topic, and helps scientific
dissemination through a definition, contributing to the development of a
theory of Agile \hl{BPM}.

Keywords: Agility, \hl{BPM} governance, Business And
Economics--Management, Business process management, challenges,
Communication, critical success factors, Customer satisfaction,
implementation capabilities, intuition, knowledge systems, Literature
reviews, operations strategy, organization, Systematic review

\hl{BPM} Perspectives to Support ICSs: Exploiting the Integration of
Formal Verifications into Investment Service Provision Processes
(@raucciBPMPerspectivesSupport2020)

Purpose This paper investigates the criteria for a selective
integration, in the multidisciplinary business process management
(\hl{BPM}) areas, between information technologies tools and the
company's internal control systems (ICSs) aimed at directing
organizational behaviours. Adopting a process-based perspective, the
authors propose a formal methodology to increase ICSs aims, related to
the segregation of duties (SoDs) models, efficiently and effectively.
Design/methodology/approach The authors examine the applicability of
formal verifications to validate a banking process of providing
investment services, which is mapped through the workflow management
system. To mitigate the state explosion problem of formal methods, the
authors propose an efficient methodology that has been proved on the
SoDs models in the bank ICSs, as a case study. Findings The authors'
investigations suggest that in the \hl{BPM} domain, the banking ICSs
aims can benefit from the aforesaid methodologies, originating from the
formal methods area, to increase the reliability and correctness in the
design, modelling and implementation of the SoDs models.
Originality/value The proposed methodology is quite general and can be
efficiently applied to large-scale systems in different business
contexts or areas of the \hl{BPM}. Its application to the bank's SoD
prevents or detects significant weaknesses, operational risks, excessive
risk appetite and other undesirable behaviours in the investment
services provision processes. This guarantees that the investment
ordered/offered is ``suitable and appropriate'\,' with the client's risk
profile, especially non-professional, required by the MiFID II
Directive.

Keywords: Banking Processes, Business Process Management, Formal
Methods, Internal Control Systems, Investment Services, Segregation of
Duties

Norms Modeling Constructs of Business Process Compliance Management
Frameworks: A Conceptual Evaluation (@hashmi2017norms)

The effectiveness of a compliance management framework (CMF) can be
guaranteed only if the framework is based on sound conceptual and formal
foundations. In particular, the formal language used in the CMF is able
to expressively represent the specifications of normative requirements
(hereafter, norms) that impose constraints on various activities of a
business process. However, if the language used lacks expressiveness and
the modelling constructs proposed in the CMF are not able to properly
represent different types of norms, it can significantly impede the
reliability of the compliance results produced by the CMF. This paper
investigates whether existing CMFs are able to provide reasoning and
modeling support for various types of normative requirements by
evaluating the conceptual foundations of the modeling constructs that
existing CMFs use to represent a specific type of norm. The evaluation
results portray somewhat a bleak picture of the state-of-the-affairs
when it comes to represent norms as none of the existing CMFs is able to
provide a comprehensive reasoning and modeling support. Also, it points
to the shortcomings of the CMFs and emphasises exigent need of new
modeling languages with sound theoretical and formal foundations for
representing legal norms.

Keywords: Business Processes, Compliance, Compliance Management
Frameworks, Modelling Constructs, Modelling Languages, Norms

RegelSpraak: A CNL for Executable Tax Rules Specification
(@corsiusRegelSpraakCNLExecutable2021)

RegelSpraak is a CNL developed at the Dutch Tax Administration (DTA)
over the last decade. Keeping up with frequently changing tax rules
poses a formidable challenge to the DTA IT department. RegelSpraak is a
central asset in ongoing efforts of the DTA to attune their tax IT
systems to automatic execution of tax law. RegelSpraak now is part of
the operational process of rule specification and execution. In this
practice-oriented paper, we present the history of RegelSpraak, its
properties and the context of its use, emphasizing its double
functionality as a language readable by non-technical tax experts but
also directly interpretable in a software generating setup.

Keywords: CNL

A Compliance Management Framework for Business Process Models
(@DBLP:phd/de/Awad2010)

Companies develop process models to explicitly describe their business
operations. In the same time, business operations, business processes,
must adhere to various types of compliance requirements.
\textbf{Regulations}, e.g., Sarbanes Oxley Act of 2002, internal
policies, best practices are just a few sources of compliance
requirements. In some cases, non-adherence to compliance requirements
makes the organization subject to legal punishment. In other cases,
non-adherence to compliance leads to loss of competitive advantage and
thus loss of market share. Unlike the classical domain-independent
behavioral correctness of business processes, compliance requirements
are domain-specific. Moreover, compliance requirements change over time.
New requirements might appear due to change in laws and adoption of new
policies. Compliance requirements are offered or enforced by different
entities that have different objectives behind these requirements.
Finally, compliance requirements might affect different aspects of
business processes, e.g., control flow and data flow. As a result, it is
infeasible to hard-code compliance checks in tools. Rather, a repeatable
process of modeling compliance rules and checking them against business
processes automatically is needed. This thesis provides a formal
approach to support process design-time compliance checking. Using
visual patterns, it is possible to model compliance requirements
concerning control flow, data flow and conditional flow rules. Each
pattern is mapped into a temporal logic formula. The thesis addresses
the problem of consistency checking among various compliance
requirements, as they might stem from divergent sources. Also, the
thesis contributes to automatically check compliance requirements
against process models using model checking. We show that extra domain
knowledge, other than expressed in compliance rules, is needed to reach
correct decisions. In case of violations, we are able to provide a
useful feedback to the user. The feedback is in the form of parts of the
process model whose execution causes the violation. In some cases, our
approach is capable of providing automated remedy of the violation.

Keywords: No keywords available

Compliance Monitoring in Business Processes: Functionalities,
Application, and Tool-Support (@lyComplianceMonitoringBusiness2015)

In recent years, monitoring the compliance of business processes with
relevant regula- tions, constraints, and rules during runtime has
evolved as major concern in literature and practice. Monitoring not only
refers to continuously observing possible compliance violations, but
also includes the ability to provide fine-grained feedback and to
predict possible compliance violations in the future. The body of
literature on business process compliance is large and approaches
specifically addressing process monitoring are hard to identify.
Moreover, proper means for the systematic comparison of these approaches
are missing. Hence, it is unclear which approaches are suitable for
particular scenarios. The goal of this paper is to define a framework
for Compliance Monitoring Functionalities (CMF) that enables the
systematic comparison of existing and new approaches for monitoring
compliance rules over business processes during runtime. To define the
scope of the framework, at first, related areas are identified and
discussed. The CMFs are harvested based on a systematic literature
review and five selected case studies. The appropriateness of the
selection of CMFs is demonstrated in two ways: (a) a systematic
comparison with pattern-based compliance approaches and (b) a
classification of existing compliance monitoring approaches using the
CMFs. Moreover, the application of the CMFs is showcased using three
existing tools that are applied to two realistic data sets. Overall, the
CMF framework provides powerful means to position existing and future
compliance monitoring approaches.

Keywords: Business process compliance, Compliance monitoring,
Operational support

Evaluation of Compliance Rule Languages for Modelling Regulatory
Compliance Requirements (@zasadaEvaluationComplianceRule2023)

Compliance in business processes has become a fundamental requirement
given the constant rise in regulatory requirements and competitive
pressures that have emerged in recent decades. While in other areas of
business process modelling and execution, considerable progress towards
automation has been made (e.g., process discovery, executable process
models), the interpretation and implementation of compliance
requirements is still a highly complex task requiring human effort and
time. To increase the level of ``mechanization'\,' when implementing
regulations in business processes, compliance research seeks to
formalize compliance requirements. Formal representations of compliance
requirements should, then, be leveraged to design correct process models
and, ideally, would also serve for the automated detection of
violations. To formally specify compliance requirements, however,
multiple process perspectives, such as control flow, data, time and
resources, have to be considered. This leads to the challenge of
representing such complex constraints which affect different process
perspectives. To this end, current approaches in business process
compliance make use of a varied set of languages. However, every
approach has been devised based on different assumptions and motivating
scenarios. In addition, these languages and their presentation usually
abstract from real-world requirements which often would imply
introducing a substantial amount of domain knowledge and interpretation,
thus hampering the evaluation of their expressiveness. This is a serious
problem, since comparisons of different formal languages based on
real-world compliance requirements are lacking, meaning that users of
such languages are not able to make informed decisions about which
language to choose. To close this gap and to establish a uniform
evaluation basis, we introduce a running example for evaluating the
expressiveness and complexity of compliance rule languages. For language
selection, we conducted a literature review. Next, we briefly introduce
and demonstrate the languages' grammars and vocabularies based on the
representation of a number of legal requirements. In doing so, we pay
attention to semantic subtleties which we evaluate by adopting a
normative classification framework which differentiates between
different deontic assignments. Finally, on top of that, we apply
Halstead's well-known metrics for calculating the relevant
characteristics of the different languages in our comparison, such as
the volume, difficulty and effort for each language. With this, we are
finally able to better understand the lexical complexity of the
languages in relation to their expressiveness. In sum, we provide a
systematic comparison of different compliance rule languages based on
real-world compliance requirements which may inform future users and
developers of these languages. Finally, we advocate for a more
user-aware development of compliance languages which should consider a
trade off between expressiveness, complexity and usability.

Keywords: business processes, compliance rules modelling, conceptual
modelling, expressiveness, language complexity, regulatory compliance

Are We Done with Business Process Compliance: State of the Art and
Challenges Ahead (@hashmiAreWeDone2018)

Literature on business process compliance (BPC) has predominantly
focused on the alignment of the regulatory rules with the design,
verification and validation of business processes. Previously surveys on
BPC have been conducted with specific context in mind; however, the
literature on BPC management research is largely sparse and does not
accumulate a detailed understanding on existing literature and related
issues faced by the domain. This survey provides a holistic view of the
literature on existing BPC management approaches, and categories them
based on different compliance management strategies in the context of
formulated research questions. A systematic literature approach is used
where search terms pertaining keywords were used to identify literature
related to the research questions from scholarly databases. From
initially 183 papers, we selected 79 papers related to the themes of
this survey published between 2000--2015. The survey results reveal that
mostly compliance management approaches center around three distinct
categories namely: design-time (28\%), run-time (32\%) and auditing
(10\%). Also, organisational and internal control based compliance
management frameworks (21\%) and hybrid approaches make (9\%) of the
surveyed approaches. Furthermore, open research challenges and gaps are
identified and discussed with respect to the compliance problem.

Keywords: Business process compliance, Business processes, Compliance
Management Frameworks, Normative requirements, Norms compliance

Comparative Analysis of Business Process Modelling Tools for Compliance
Management Support (@DBLP:journals/rigaacs/KoncevicsPGDBA17)

The paper presents results of the comparative analysis of business
process modelling tools for supporting automated compliance management
in organisations. By \textbf{compliance} in the paper we mean compliance
to legislation, \textbf{regulations} of municipalities, external
regulatory requirements and also internal organisational policies. The
goal of the research is (1) to identify main attributes of business
process modelling tools relevant in compliance management, and (2) to
use the identified attributes for analysis of the tools to better
understand the scope of their capability to support compliance
management. The attributes of the tools have been derived from the
related research. The analysis of the tools has been performed by
installing each tool and evaluating it against a set of the identified
attributes. The obtained results are useful in choosing the tools for
compliance management in general and for open source solutions to
develop new compliance management tools in particular.

Keywords: Business process compliance, compliance management, compliance
management tools, open source business process modelling tools

Blockchain-Based Business Process Management (\hl{BPM}) Framework for
Service Composition in Industry 4.0
(@viriyasitavatBlockchainbasedBusinessProcess2020)

Business process management (\hl{BPM}) aims to optimize business
processes to achieve better system performance such as higher profit,
quicker response, and better services. \hl{BPM} systems in Industry 4.0
are required to digitize and automate business process workflows and
support the transparent interoperations of service vendors. The critical
bottleneck to advance \hl{BPM} systems is the evaluation, verification,
and transformation of trustworthiness and digitized assets. Most of
\hl{BPM} systems rely heavily on domain experts or third parties to deal
with trustworthiness. In this paper, an automated \hl{BPM} solution is
investigated to select and compose services in open business
environment, Blockchain technology (BCT) is explored and proposed to
transfer and verify the trustiness of businesses and partners, and a
\hl{BPM} framework is developed to illustrate how BCT can be integrated
to support prompt, reliable, and cost-effective evaluation and
transferring of Quality of Services in the workflow composition and
management.

Keywords: Block-chain technology (BCT), Business process management
(\hl{BPM}), Industry 4.0, Internet of Things (IoT), Quality of Service
(QoS), Service selection and composition, Smart contracts,
Trustworthiness

Comparing Textual Descriptions to Process Models -- The Automatic
Detection of Inconsistencies (@vanderaaComparingTextualDescriptions2017)

Many organizations maintain textual process descriptions alongside
graphical process models. The purpose is to make process information
accessible to various stakeholders, including those who are not familiar
with reading and interpreting the complex execution logic of process
models. Despite this merit, there is a clear risk that model and text
become misaligned when changes are not applied to both descriptions
consistently. For organizations with hundreds of different processes,
the effort required to identify and clear up such conflicts is
considerable. To support organizations in keeping their process
descriptions consistent, we present an approach to automatically
identify inconsistencies between a process model and a corresponding
textual description. Our approach detects cases where the two process
representations describe activities in different orders and detect
process model activities not contained in the textual description. A
quantitative evaluation with 53 real-life model-text pairs demonstrates
that our approach accurately identifies inconsistencies between model
and text.

Keywords: Business process management, Business process modeling,
Compliance checking, Inconsistency detection, Matching, Natural language
processing

Blockchains for Business Process Management-Challenges and Opportunities
(@mendlingBlockchainsBusinessProcess2018)

Blockchain technology promises a sizable potential for executing
inter-organizational business processes without requiring a central
party serving as a single point of trust (and failure). This paper
analyzes its impact on business process management (\hl{BPM}). We
structure the discussion using two \hl{BPM} frameworks, namely the six
\hl{BPM} core capabilities and the \hl{BPM} lifecycle. This paper
provides research directions for investigating the application of
blockchain technology to \hl{BPM}.

Keywords: Top100

Business Process Reengineering of Emergency Management Procedures: A
Case Study (@bevilacquaBusinessProcessReengineering2012)

The production and storage of dangerous substances in an industrial
establishment creates risks for man, environment and properties in the
surrounding area. Safety regulations require the establishment of a
preventive information campaign regarding industrial risks and
self-defence measures to adopt in an emergency situation. In the case of
a major accident, people must be promptly made aware of the appropriate
self-defence actions and behaviours to adopt. This strategic activity
can reduce the panic effect, make citizens more cooperative and
guarantee the effectiveness of any emergency plan. In this paper, the
information chain is studied as an industrial process modelled by the
IDEF0 language. Through this method, each link in the chain has been
deeply analysed. For each function of the process, the inputs, outputs
and necessary controls and resources have been identified. Starting from
a clear view of the current state, the process of re-engineering has
been implemented to minimise or eliminate downtime, deficiencies and
illnesses and, thus, consequent time losses. The main contribution of
the IDEF0 application in emergency management is to provide a clear view
of the whole system, a communication system between emergency actors, a
rich information source and a structured base for the re-engineering
process.

Keywords: Emergency management, IDEF0, Information supply, Information
system, Public, Risk information, Safety management

Product-Based Workflow Support
(@vanderfeestenProductbasedWorkflowSupport2011)

Despite the industrial need for the improvement of information-intensive
business processes, few scientifically grounded approaches exist to
support such initiatives. In this paper, we propose a new approach that
builds on concepts that are part of a product-oriented view on process
optimization. Essentially, this approach allows end users to flexibly
decide on the best possible way to create an informational product
within the limits that are imposed by regulations and logical
dependencies. We argue that this provides various benefits in comparison
to earlier work. To support the end user in making sensible decisions,
we describe two alternative approaches to provide her with
recommendations to this end. We formalize these alternatives and discuss
their relative strengths and weaknesses. The feasibility of the overall
approach, which we refer to as Product-Based Workflow Support, is
demonstrated by a workflow system realized using ProM and DECLARE.

Keywords: Business Process Modelling, Product Data Model, Workflow
Management

Recognizing and Splitting Conditional Sentences for Automation of
Business Processes Management (@voRecognizingSplittingConditional2021)

Business Process Management (\hl{BPM}) is the discipline which is
responsible for management of discovering, analyzing, redesigning,
monitoring, and controlling business processes. One of the most crucial
tasks of \hl{BPM} is discovering and modelling business processes from
text documents. In this paper, we present our system that resolves an
end-to-end problem consisting of 1) recognizing conditional sentences
from technical documents, 2) finding boundaries to extract conditional
and resultant clauses from each conditional sentence, and 3)
categorizing resultant clause as Action or Consequence which later helps
to generate new steps in our business process model automatically. We
created a new dataset and three models solve this problem. Our best
model achieved very promising results of 83.82, 87.84, and 85.75 for
Precision, Recall, and F1, respectively, for extracting Condition,
Action, and Consequence clauses using Exact Match metric.

Keywords: Computer Science - Computation and Language

Spreadsheets for Business Process Management: Using Process Mining to
Deal with
\texttt{Events\textquotesingle{}\textquotesingle{}\ Rather\ than}Numbers'\,'?
(@vanderaalstSpreadsheetsBusinessProcess2018)

Purpose -- Process mining provides a generic collection of techniques to
turn event data into valuable insights, improvement ideas, predictions,
and recommendations. This paper uses spreadsheets as a metaphor to
introduce process mining as an essential tool for data scientists and
business analysts. The purpose of this paper is to illustrate that
process mining can do with events what spreadsheets can do with numbers.

Keywords: No keywords available

Supporting Domain Experts to Select and Configure Precise Compliance
Rules (@DBLP:conf/\hl{BPM}/RamezaniFA13)

Compliance specifications concisely describe selected aspects of what a
business operation should adhere to. To enable automated techniques for
compliance checking, it is important that these requirements are
specified correctly and precisely, describing exactly the behavior
intended. Although there are rigorous mathematical formalisms for
representing compliance rules, these are often perceived to be difficult
to use for business users. Regardless of notation, however, there are
often subtle but important details in compliance requirements that need
to be considered. The main challenge in compliance checking is to bridge
the gap between informal description and a precise specification of all
requirements. In this paper, we present an approach which aims to
facilitate creating and understanding formal compliance requirements by
providing configurable templates that capture these details as options
for commonly-required compliance requirements. These options are
configured interactively with end-users, using question trees and
natural language. The approach is implemented in the Process Mining
Toolkit ProM.

Keywords: auditing, compliance checking, compliance specification,
configurable compliance rules, question tree

A Knowledge-Intensive Adaptive Business Process Management Framework
(@kirKnowledgeintensiveAdaptiveBusiness2021)

Business process management has been the driving force of optimization
and operational efficiency for companies until now, but the
digitalization era we have been experiencing requires businesses to be
agile and responsive as well. In order to be a part of this digital
transformation, delivering new levels of automation-fueled agility
through digitalization of \hl{BPM} itself is required. However, the
automation of \hl{BPM} cannot be achieved by solely focusing on process
space and classical planning techniques. It requires a holistic approach
that also captures the social aspects of the business environment, such
as corporate strategies, organization policies, negotiations, and
cooperation. For this purpose, we combine \hl{BPM}, knowledge-intensive
systems and intelligent agent technologies, and yield one consolidated
intelligent business process management framework, namely agileBPM, that
governs the entire \hl{BPM} life-cycle. Accordingly, agileBPM proposes a
modeling methodology to semantically capture the business interests,
enterprise environment and process space in accordance with the
agent-oriented software engineering paradigm. The proposed agent-based
process execution environment provides cognitive capabilities (such as
goal-driven planning, norm compliance, knowledge-driven actions, and
dynamic cooperation) on top of the developed business models to support
knowledge workers' multi-criteria decision making tasks. The context
awareness and exception handling capabilities of the proposed approach
have been presented with experimental studies. Through comparative
evaluations, it is shown that agileBPM is the most comprehensive
knowledge-intensive process management solution.

Keywords: Agent-based business process management, Agile business
process management, Business process management, Knowledge-intensive
processes, Process adaptation, Process modeling and execution

Improved Compliance by \hl{BPM}-Driven Workflow Automation
(@holzmuller-laueImprovedComplianceBPMDriven2014)

Using methods and technologies of business process management (\hl{BPM})
for the laboratory automation has important benefits (i.e., the agility
of high-level automation processes, rapid interdisciplinary prototyping
and implementation of laboratory tasks and procedures, and efficient
real-time process documentation). A principal goal of the model-driven
development is the improved transparency of processes and the alignment
of process diagrams and technical code. First experiences of using the
business process model and notation (BPMN) show that easy-to-read
graphical process models can achieve and provide standardization of
laboratory workflows. The model-based development allows one to change
processes quickly and an easy adaption to changing requirements. The
process models are able to host work procedures and their scheduling in
compliance with predefined guidelines and policies. Finally, the
process-controlled documentation of complex workflow results addresses
modern laboratory needs of quality assurance. BPMN 2.0 as an automation
language to control every kind of activity or subprocess is directed to
complete workflows in end-to-end relationships. BPMN is applicable as a
system-independent and cross-disciplinary graphical language to document
all methods in laboratories (i.e., screening procedures or analytical
processes). That means, with the \hl{BPM} standard, a communication
method of sharing process knowledge of laboratories is also available.

Keywords: BPMN, end-to-end workflow, laboratory automation, model-based
application development, systems integration

Business Process Compliance Management: An Integrated Proactive Approach
(@elgammalBusinessProcessCompliance2014)

Today's enterprises demand a high degree of compliance of business
processes to meet regulations, such as Sarbanes-Oxley and Basel I-III.
To ensure continuous guaranteed compliance, compliance management should
be considered during all phases of the business process lifecycle; from
the analysis and design to deployment, monitoring and evaluation. This
paper introduces an integrated business process compliance management
framework that incorporates design-time verification and runtime
monitoring approaches. The nutshell of the approach is the Compliance
Request Language (CRL), which is a high-level pattern-based language for
the abstract specification of compliance requirements. From CRL
expressions, formal compliance rules can be automatically generated,
thereby eliminating the need for business and compliance experts to
learn and use complex low-level formal languages. Formalized compliance
rules enable automated approaches to be used for the static verification
and dynamic monitoring of business processes. An integrated prototypical
tool-suite is developed as a proof-of-concept to help validating the
applicability of the approaches, and validated by experiment with two
real-life case studies.

Keywords: No keywords available

The Influence of Directive Explanations on Users' Business Process
Compliance Performance (@hadaschInfluenceDirectiveExplanations2016)

Purpose -- In organizations, individual user's compliance with business
processes is important from a regulatory and efficiency point of view.
The restriction of users' choices by implementing a restrictive
information system is a typical approach in many organizations. However,
restrictions and mandated compliance may affect employees' performance
negatively. Especially when users need a certain degree of flexibility
in completing their work activity. The purpose of this paper is to
introduce the concept of directive explanations (DEs). DEs provide
context-dependent feedback to users, but do not force users to comply.

Keywords: No keywords available



\end{document}
